
\documentclass[11pt,secnumarabic,nofootinbib,preprintnumbers,amsmath,amssymb,aps]{revtex4}
\usepackage[usenames]{color}
\usepackage{graphicx}
\usepackage{ulem}
%\graphicspath{plots\\}

\frenchspacing
\def\red{\textcolor{red}}
\def\blue{\textcolor{blue}}


%%%%%

\begin{document}
  \title{Scalar Field Dark Matter}

 % \author{J. Alberto V\'azquez$^{1}$.}
 % \email{jvazquez@bnl.gov}
%   \author{M.P. Slozar$^2$}
%  \affiliation{$^1$Kavli Institute for Cosmology, Madingley Road, Cambridge CB3 0HA, UK. }
  % \date{\today}


%%=================================================================%%%
%\begin{abstract}
%%=================================================================%%%



%\end{abstract}

%\keywords{latex-community, revtex4, aps, papers}
\maketitle
%
%%=================================================================%%%
%\section{}
%%=================================================================%%%
%

\section{Proyectos de SFDM}

\begin{enumerate}
\item Por ejemplo, Constrinir la masa del campo escalar usando el Codigo de Urena que contiene 
las perturbaciones.

\item Con este mismo codigo calcular el Matter power Spectrum P(k), a escalar pequenas $k\sim 1$ y
redshift grandes $z\sim 2.5$ (Lyman-alpha) y compararlo con los calculos de Abril.

  \begin{figure}[h!]
      \label{fig:Pk}
 \begin{center}
   \includegraphics[trim = 2mm 1mm 1mm 1mm, clip, width=7cm, height=5cm]{Matter.pdf}
  \caption{}
 \end{center}
\end{figure} 

\item Sistemas din\'amicos (escribir rutinas de ABM en Python) para crear una librer�a que cualquiera
pueda usar.

\item Bispectrum

\item Missing Halos

\item Lensing

\item Simulaciones Numericas


\end{enumerate}

\end{document}





















